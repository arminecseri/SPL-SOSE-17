
\section{Introduction}




Needless to say, that in our current society real estates play a quintessential part of our everyday lives. The first thing that comes to our minds when thinking about investing into real estates and properties is home. For most of the people on earth, buying a home will be the single largest investment they will ever make. But apart from buying one home for ourselves, and keeping it, we can also buy one, wait until the prices go up, and sell the home later, making a profit.\\
Estimating the value of real property is necessary for a variety of endeavors, including real estate financing, listing real estate for sale, investment analysis, property insurance and the taxation of real estate. For most people, determining the asking or purchase price of a property is the most useful application of real estate valuation.\\
In this paper, we are trying to find out, how the price of a real estate is constructed.\\
We are looking for relationships between the price, and other fundamental attributes of a house, such as the size, bedrooms, bathrooms, garage capacity, etc. What is the strongest influence on the price? Is it the number of bedrooms, or the size of the house? Is it true that the bigger a house is, the more it should cost? How does the age of the building affects the price of a property? Can we make the general conclusion that the newer the building, the higher the price should be? Or maybe people think that older, really nice classical buildings have a higher value, and that?s why agencies might put a higher price on them?  These are the questions we are answering in this paper with the help of the programming language R.\\

In sections 2 and 3 we are describing the data, and the modifications that we used for our analysis. It?s important to understand for the readers why and how we adjusted and tidied the data. Then in section 4, we carry out a descriptive analysis about the dataset that we used, to get a closer look about the two states and the cities in the dataset, and how the observations were distributed. In section 5, we are doing an exploratory analysis. Firstly, we are introducing quantile-quantile plotting, and correlation analysis, to prepare the data for the regression models. We assume that the correlations between the price and fundamental variables like the size, number of bedrooms, and rooms in total will have a high correlation. Then we answer our most important questions with different statistical methods, dividing the dataset into a test set and a train set. Our methods include simple linear regression, multivariate regression, and multivariate robust regression. Concluding in section 6, we briefly discuss the errors and problems we faced, and summarize our analysis.\\
